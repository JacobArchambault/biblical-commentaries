\chapter{Romans 1}
\section{Paul's salutation (1-7)}
\begin{quote}
	Paul, servant of Jesus the anointed, called missionary, separated out unto the glad tidings of God, which he announced before by the holy prophets in holy writings, concerning his son, the one brought forth from the seed of David after the flesh, the one destined the son of God in power after the spirit of holiness from the rising again from the dead, Jesus the anointed, our lord, by whom we have taken grace and the mission away, unto the hearing of the faith in all the nations for his name, in which you, too, are, called of Jesus the anointed, to all the beloved of God, called holy, who are in Rome: grace to you, and peace from the God of our fathers and the lord Jesus the anointed.
\end{quote}

Paul begins his letter first by 1) introducing himself, 2) then his subject, before 3) naming those to whom the letter is addressed and 4) calling a blessing upon them. 

\subsection{Who Paul is}
Paul introduces himself by three descriptions: \emph{servant of Jesus the anointed}, one \emph{called missionary} or apostle, one \emph{separated out unto the good message of God}. Each description serves a distinct purpose. 
\begin{itemize}
	\item The first indicates Paul's submission to Jesus as his master, contrasted with the local image of Paul as one seeking to raise himself up by his own novel teaching. 
	\item The second gives the title by which he was known, and thereby makes claim to his vocation being of the same nature as that of Peter, James, and the others whose teaching was accepted there.
	\item The third gives a consequence of the second, namely that by being an apostle, Paul has become one set apart: separated in holiness, but also cut off from his own people and native land, for the purpose of God's good news. 
\end{itemize}


\subsection{Who Jesus is}
Having introduced himself, Paul then introduces his subject, first, generally, then specifically under several descriptions. 

Paul describes his subject generally under two descriptions: that of good news, good message, or glad tidings, and that of what God announced before through the holy prophets in holy writings. These descriptions contrast with each other: the first connotes the message as newly recognized; the second, as from of old. 

