\chapter{Romans 1: 1-7}
\section{Paul's salutation (1-7)}
\begin{quote}
	Paul, servant of Jesus the anointed, called missionary, separated out unto the glad tidings of God, which he announced before by the holy prophets in holy writings, concerning his son, the one brought forth from the seed of David after the flesh, the one destined the son of God in power after the spirit of holiness from the rising again from the dead, Jesus the anointed, our lord, by whom we have received grace and the mission away, unto the hearing of the faith in all the nations for his name, in which you, too, are, called of Jesus the anointed, to all those who are in Rome, the beloved of God, called holy: grace to you, and peace from God the father of us and of the lord Jesus the anointed.
\end{quote}

Paul begins his letter first by 1) introducing himself, 2) then his subject, before 3) stating the mission common to himself and his letter's recipients, 4) naming those to whom the letter is addressed and 5) calling a blessing upon them. 

\section{Who Paul is}
Paul introduces himself by three descriptions: \emph{servant of Jesus the anointed}, one \emph{called missionary} or apostle, one \emph{separated out unto the good message of God}. Each description serves a distinct purpose. 
\begin{itemize}
	\item The first indicates Paul's submission to Jesus as his master, contrasted with the local image of Paul as one seeking to raise himself up by his own novel teaching. 
	\item The second gives the title by which he was known, and thereby makes claim to his vocation being of the same nature as that of Peter, James, and the others whose teaching was accepted there.
	\item The third gives a consequence of the second, namely that by being an apostle, Paul has become one set apart: separated in holiness, but also cut off from his own people and native land, for the purpose of God's good news. 
\end{itemize}


\section{Who Jesus is}
Having introduced himself, Paul then introduces his subject, first, generally, then specifically under several descriptions. 

Paul describes his subject generally under two descriptions: that of good news, good message, or glad tidings, and that of what God announced before through the holy prophets in holy writings. These descriptions contrast with each other: the first connotes the newness of the message's reception; the second, the antiquity of its origin. 

The specific description of Paul's subject as \emph{Jesus} begins by contrasting his carnal origin with his divine destiny: Jesus is brought forth from the seed of David, but ordained the son of God; the first, from his being David's seed \emph{after the flesh}; the second, from his rising again \emph{after the spirit of holiness}. In drawing this contrast, Paul confirms Jesus' birthright to the earthly throne of David while simultaneously casting it as a lesser good than the divine birthright he receives from God himself in his rising again.\footnote{In his own preaching, Jesus himself makes a similar point via the example of John the baptist (Mt 11:11, Lk 7:28).}

Only after introducing Jesus under the description of his earthly and divine claims does Paul mention him by name. He immediately follows with two descriptions: \emph{anointed} or messiah, indicating the divine source of his authority and his announcement from of old to God's people, and \emph{lord}, indicating his relation to God's people as leader, example and master. 


\section{Paul and the church at Rome's common constitution and purpose}
Paul then states the common constitution and purpose between himself and his hearers. In and through these dual roles of Jesus as messiah and lord, all God's people has received both a) \emph{grace}: that is, spiritedness, clemency, and favor, and b) the mission out, or apostleship, that is, its separation from the world in holiness and its commission to witness to it.\footnote{In extending the latter title to all under God's favor, he further undercuts any temptation to regard the divine commission as belonging solely or principally to those to whom the term `apostle' was accorded as a formal title.} The purpose of this anointing and mission is \emph{unto the obedience of faith in all nations for his name}. Here the word for `obedience', literally `under-hearing', connotes submission. The sense can be understood in two ways. 
\begin{itemize}
	\item In the first, the faith, that is, the word which Paul preaches, is to be given a hearing: put on trial and heard out, so as to be vindicated.
	\item In the second, the faith, upon being heard out, is to be taken to heart, obeyed, and submitted to among all peoples.
\end{itemize}

In both cases, `faith' refers to the content of Paul's preaching and God's promise to his people through the risen messiah. In no case does `faith' here refer to the passionate conviction of belief of either an ecclesial body or an individual believer. Further, when Paul speaks about the obedience of faith being found among all nations, he does not say the obedience of faith \emph{of} all nations, but {in} all nations. Consequently, he is not indicating the formal conversion of entire nations to the Christian faith, but God's gathering a people set apart in and out of each nation. He ends by predicating the aforementioned grace and apostleship of his letter's recipients: \emph{in which you, too, are}.

\section{The church of Rome}
Paul addresses the recipients of his letter first by their location, then by their relation to God: \emph{to all those who are in Rome, called of Jesus the anointed, the beloved of God, called holy}. In the first phrase, `all' does not indicate all Romans in the city, but all \emph{called of Jesus the anointed} - more colloquially, all those called Christians, as Jesus' followers first were at Antioch. These are beloved of God, called holy. The text does not say that they are called \emph{to be} holy, but that they are recognized as such. Paul ends his salutation with a blessing: \emph{grace}, that is, God's charity and favor \emph{to you, and peace from God the father of us and of the lord Jesus the anointed.} the \emph{from God our father...} is ambiguous: on the one hand, it can be read as `peace from God our father, and from the lord Jesus'; on the other, as `peace from God the father of us and of the lord Jesus'. The first formulation is the most commonly followed, having already been adopted unambiguously in the Latin text and later in the English translations of Wyclif, the Douay Rheims, and the King James versions, and drawing theological support for ascribing the peace offering equally to both from traditional Christology's placement of the son as co-equal with the father. But the latter reading accords better with the theme of the epistle up to this point: Jesus, though son of David according to the flesh, is God's son by the power of the spirit manifested in his rising again. Likewise, those called `Christians' after the anointed messiah whose promised coming they confide in, to whom this letter is addressed, can look forward in faith to their own resurrection as God's children on the confidence of God's promise.

\section{Verses 8-17}
Now first, I give grace well to my God by Jesus the anointed concerning all of you, that your faith is announced forth in the whole world. For my witness is God, to whom I serve in my spirit in the good message of his son, that I make your remembrance unceasingly, always upon my prayers setting apart (if by any way some time), that I may be given in God's will to go for you. For I desire to see you, that may give over some spiritual grace to you unto making you firm - which is, however, to be strengthened together among you, by faith, both yours and mine, among each other. But I do not wish you to not know, brothers, that I have often set forth to go for you -  and I am barred even now - that some fruit I might have, too, among you, just as also in the other nations. For  to both Greeks and barbarians, wise and ignorant I am indebted, such that after me is urged forth to announce well to you also in Rome. For I do not blush at the good message. for the power of God is unto healing to each believing, both Jew first and Greek. For God's justness in him is unhidden, from faith unto faith, just as it is written, the just one shall live from faith. 

\section{}
Paul begins his letter proper with an offering of thanks, succesively naming its recipient, occasion, reason, and witness.